\documentclass[14pt]{article}
\usepackage[margin=1in]{geometry}
\usepackage{amsmath, amssymb, mathtools}
\usepackage{tikz}
\usepackage{lmodern}
\usepackage{hyperref}
\usepackage{caption}
\usepackage{float}
\usepackage{parskip}
\usepackage{tabularx}
\usepackage[utf8]{inputenc}
\usepackage{tgpagella}
\usepackage[T1]{fontenc}
\usepackage{array} % For better table features
\usepackage[utf8]{inputenc}
\usepackage[margin=1in]{geometry}
\usepackage{booktabs}
\usepackage{tcolorbox}
\usepackage{enumitem}

\title{Financial Statement Analysis}
\author{}
\date{}

\begin{document}

\maketitle

\tableofcontents

\newpage

\noindent\rule{\linewidth}{1pt}
%-----------------------------------------------------------------

\section{Introduction to Financial Statement Analysis}
How do we determine if a company is doing well financially? For example, Amazon had a net income of \$0.6 billion at the end of 2015. Is that good or bad? To answer this, we look at financial ratios.

In this video, we will introduce the idea of financial statement analysis. We will also talk about the different tools used in financial statement analysis, as well as the approaches used with each tool. We will wrap up this video by defining the 5 most common categories of financial ratios.

\textbf{Financial Statement Analysis} is a comprehensive analysis of:
\begin{itemize}
    \item A company's strategy.
    \item Its competition, regulation, and factors affecting it.
    \item Its past and current financial performance.
    \item The fundamental evaluation of a company relative to its stock price.
    \item Planning for the company's future operations, investments, and finances.
\end{itemize}

\textit{Note: In this course, we focus only on financial performance and fundamental evaluation relative to stock price.}

\subsection{Tools of Financial Statement Analysis}
The typical tools used are:

\subsubsection{Comparative Analysis}
The evaluation of consecutive financial statements of a company to identify the direction, speed, and magnitude of any trends in financial performance.

\subsubsection{Common Size Analysis}
The evaluation of the internal make-up of financial statements and/or financial statement items across companies.
\begin{itemize}
    \item \textbf{Income Statement:} All items reported as a percentage of revenues. This is useful when comparing companies with widely different revenues.
    \item \textbf{Balance Sheet:} All items reported as a percentage of total assets.
\end{itemize}

\subsubsection{Ratio Analysis}
Evaluates the relationship between two or more economically important items. Prior accounting analysis and interpretation are very important.
\textit{This course focuses mostly on ratio analysis.}

\subsection{Approaches to Analysis}
The approaches used with the tools above are:

\begin{itemize}
    \item \textbf{Time-Series Analysis:} Comparison of a firm with itself over time. (Focus of this course).
    \item \textbf{Cross-Sectional Analysis:} Comparison of different companies in an industry at a given point in time.
    \item \textbf{Benchmark Comparison:} Use of pre-specified industry norms or benchmarks.
\end{itemize}

\subsection{Categories of Financial Ratios}
Financial ratios largely fall into five categories:

\begin{enumerate}
    \item \textbf{Profitability Ratios:} Measure a company's ability to generate profits from its various resources.
    \item \textbf{Activity Ratios:} Measure a company's ability to convert various assets and liabilities into cash or savings.
    \item \textbf{Solvency Ratios:} Measure if a company has sufficient cash to meet its long-term financial commitments.
    \item \textbf{Liquidity Ratios:} Measure if a company has enough cash to meet its short-term financial obligations.
    \item \textbf{Valuation Ratios:} Compare the current market price of a company stock to certain items from the financial statements.
\end{enumerate}

Next time we will start looking at the various types of ratios within each category of financial ratios.

\end{document}