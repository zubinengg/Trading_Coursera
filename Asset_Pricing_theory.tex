\documentclass[12pt]{article}
\usepackage[margin=1in]{geometry}
\usepackage{amsmath, amssymb, mathtools}
\usepackage{tikz}
\usepackage{lmodern}
\usepackage{hyperref}
\usepackage{caption}
\usepackage{float}
\usepackage{parskip}
\usepackage{tabularx}
\usepackage[utf8]{inputenc}
\usepackage{tgpagella}
\usepackage[T1]{fontenc}
\usepackage{array} % For better table features
\usepackage{booktabs}
\usepackage{tcolorbox}
\usepackage{enumitem}

\title{Asset Pricing Theory}
\author{Zubin}
\date{\today}

\begin{document}

\maketitle

\tableofcontents

\newpage

\noindent\rule{\linewidth}{1pt}
%-----------------------------------------------------------------

\section{Expected Returns and Risk}

\subsection{Holding Period Return}
The holding period return measures the total return earned on an investment over a specific period. It consists of two components:
\begin{enumerate}
    \item \textbf{Income:} Dividends or interest received.
    \item \textbf{Capital Gain/Loss:} The change in the price of the asset.
\end{enumerate}

The general formula for Holding Period Return (HPR) is:
\[
    R = \frac{P_1 - P_0 + D_1}{P_0}
\]
Where:
\begin{itemize}
    \item $P_0$: Initial Price (Purchase Price)
    \item $P_1$: Price at the end of the period
    \item $D_1$: Dividend paid during the period
\end{itemize}

\textbf{Example:}
You buy a stock for \$50 ($P_0$). After one year, the price is \$55 ($P_1$) and it paid a dividend of \$1 ($D_1$).
\[
    R = \frac{55 - 50 + 1}{50} = \frac{6}{50} = 0.12 \text{ or } 12\%
\]

\subsection{Expected Return and Risk}
In reality, future prices ($P_1$) and dividends ($D_1$) are unknown. Therefore, the return is a \textbf{random variable} characterized by possible outcomes and their probabilities.

\begin{itemize}
    \item \textbf{Expected Return ($E[R]$):} The probability-weighted average of all possible returns. It is used as the discount rate for present value calculations.
    \item \textbf{Risk:} The uncertainty of future returns (both good and bad outcomes). It is typically measured by the \textbf{Variance} ($\sigma^2$) or \textbf{Standard Deviation} ($\sigma$).
\end{itemize}

\subsection{Comparison: Risk-Free vs. Risky Investment}
Consider an investor with \$100 and two investment choices:

\subsubsection{Investment 1: Risk-Free}
Guaranteed payoff of \$105 after 1 year.
\begin{itemize}
    \item \textbf{Return:} $\frac{105 - 100}{100} = 5\%$
    \item \textbf{Risk:} Since there is no uncertainty, Risk ($\sigma$) = 0.
\end{itemize}

\subsubsection{Investment 2: Risky}
Payoff depends on probabilities:
\begin{itemize}
    \item 60\% probability of \$150 (Return: $\frac{150-100}{100} = 50\%$)
    \item 40\% probability of \$80 (Return: $\frac{80-100}{100} = -20\%$)
\end{itemize}

\textbf{Calculating Expected Return:}
\[
    E[R] = (0.6 \times 0.50) + (0.4 \times -0.20) = 0.30 - 0.08 = 0.22 \text{ or } 22\%
\]

\textbf{Calculating Risk (Variance and Standard Deviation):}
Variance is the weighted average of squared deviations from the expected return.
\[
    \sigma^2 = \sum P_i (r_i - E[R])^2
\]
\[
    \sigma^2 = 0.6(0.50 - 0.22)^2 + 0.4(-0.20 - 0.22)^2
\]
\[
    \sigma^2 = 0.6(0.28)^2 + 0.4(-0.42)^2 = 0.6(0.0784) + 0.4(0.1764)
\]
\[
    \sigma^2 = 0.04704 + 0.07056 = 0.1176
\]
\[
    \text{Standard Deviation } (\sigma) = \sqrt{0.1176} \approx 34.29\%
\]

\subsubsection{Conclusion}
Investment 2 offers a higher expected return (22\% vs 5\%) but comes with significantly higher risk (34.29\% vs 0\%). Whether the additional 17\% return is sufficient compensation for the risk depends on the investor's \textbf{utility} (attitude toward risk) and \textbf{asset pricing models}.

\noindent\rule{\linewidth}{1pt}
%-----------------------------------------------------------------

\section{Diversification and Risk}

\subsection{The Concept of Diversification}
Investing in a single stock is risky; if the company fails, the investor loses their entire investment. To mitigate this, investors form a \textbf{portfolio} (a collection of stocks).
\begin{itemize}
    \item \textbf{Diversification:} The strategy of investing in multiple assets to spread risk.
    \item \textbf{Mechanism:} In a portfolio, the poor performance of some stocks is likely to be offset by the good performance of others. Therefore, the risk (variance) of a portfolio is typically lower than the risk of a single stock.
\end{itemize}

\subsection{Types of Risk}
As the number of stocks in a portfolio increases, the portfolio variance (risk) decreases. This relationship reveals two distinct types of risk:

\begin{enumerate}
    \item \textbf{Diversifiable Risk (Unsystematic Risk):}
          \begin{itemize}
              \item The portion of risk that can be eliminated by adding more stocks to the portfolio.
              \item It is represented by the gap between the total risk curve and the minimum risk floor.
              \item This risk becomes negligible with a portfolio of about 50 to 60 stocks.
          \end{itemize}
    \item \textbf{Systematic Risk:}
          \begin{itemize}
              \item The portion of risk that \textbf{cannot} be eliminated, regardless of how many stocks are added.
              \item It represents the inherent uncertainty of the market that affects all stocks.
          \end{itemize}
\end{enumerate}

\subsection{Compensation for Risk}
A fundamental principle is that investors should \textbf{not} expect compensation (higher returns) for holding diversifiable risk. Since this risk can be eliminated for free through diversification, the market does not reward it. Investors are compensated only for holding systematic risk.

\subsubsection{Arbitrage Example}
Consider a scenario with a risk-free rate of 5\% and four stocks (A, B, C, D) that have \textbf{no systematic risk} (only diversifiable risk).
\begin{itemize}
    \item Expected Returns: A (12\%), B (15\%), C (17\%), D (20\%).
\end{itemize}
If an investor forms an equally weighted portfolio (\$25 in each):
\[
    E[R_p] = 0.25(12\%) + 0.25(15\%) + 0.25(17\%) + 0.25(20\%) = 16\%
\]
\textbf{Analysis:} Since the stocks have no systematic risk, a diversified portfolio eliminates the diversifiable risk, resulting in a total risk of zero. This creates a risk-free asset yielding 16\%, while the market risk-free rate is only 5\%. Investors would borrow at 5\% and invest in this portfolio to earn a guaranteed profit (Arbitrage). This demand would drive up stock prices, reducing returns until they equal the risk-free rate (5\%).

\textbf{Conclusion:} In the long run, investors are not compensated for diversifiable risk.

\noindent\rule{\linewidth}{1pt}
%-----------------------------------------------------------------
\section{The Capital Asset Pricing Model (CAPM)}

\subsection{The CAPM Formula}
The Capital Asset Pricing Model (CAPM) defines the mathematical relationship between expected return and risk. The formula is written as:
\[
    E[r_i] = r_f + \beta_i (E[r_m] - r_f)
\]
Where:
\begin{itemize}
    \item $E[r_i]$: Expected return of risky asset $i$.
    \item $r_f$: Risk-free rate of return.
    \item $\beta_i$: Beta of the risky asset $i$.
    \item $E[r_m]$: Expected return of the market portfolio.
\end{itemize}

In plain English, the expected return of a risky asset over and above the risk-free rate equals the sensitivity of its return with respect to the market ($\beta$) times the expected market risk premium.

\subsection{Market Risk Premium}
The term $(E[r_m] - r_f)$ is the \textbf{Expected Market Risk Premium}. It represents the return in excess of the risk-free rate that an investor expects to earn from holding the market portfolio. It is the compensation for holding the risk of the market portfolio.

\subsection{Beta ($\beta$) and Systematic Risk}
Since the market portfolio is mean-variance efficient, all diversifiable (idiosyncratic) risk has been diversified away. The risk that remains is \textbf{Systematic Risk}.
\begin{itemize}
    \item Beta is a measure of systematic risk.
    \item Investors are compensated \textbf{only} for holding systematic risk, not diversifiable risk.
\end{itemize}

\subsubsection{Calculating Beta}
Beta is calculated as the covariance between the risky asset's returns and the market portfolio's returns, divided by the variance of the market portfolio's returns:
\[
    \beta_i = \frac{Cov(r_i, r_m)}{Var(r_m)} = \frac{Cov(r_i, r_m)}{\sigma_m^2}
\]

\subsubsection{Interpreting Beta Values}
\begin{itemize}
    \item $\beta = 1$: The beta of the market portfolio is 1.
    \item $\beta > 1$: The asset is riskier than the market portfolio.
    \item $\beta < 1$: The asset is less risky than the market portfolio.
    \item $\beta = 0$: The risk-free asset has a beta of 0.
    \item $\beta < 0$: A negative beta means the asset's return moves in the opposite direction to the market's return.
\end{itemize}

\noindent\rule{\linewidth}{1pt}
%-----------------------------------------------------------------

\section{Calculating CAPM Beta}
We now apply the CAPM formula to calculate the betas of risky assets.

\subsection{Scenario Setup}
Let's revisit our three risky asset case (X, Y, Z) and one risk-free asset ($R_f = 5\%$). In previous examples, we found that some portfolios had negative weights, which cannot be a market portfolio. To solve this, we adjust the covariances slightly to ensure positive weights.

\begin{itemize}
    \item \textbf{Asset X:} $E[R] = 10\%$, $\sigma = 7\%$
    \item \textbf{Asset Y:} $E[R] = 20\%$, $\sigma = 10\%$
    \item \textbf{Asset Z:} $E[R] = 15\%$, $\sigma = 12\%$
\end{itemize}

\begin{table}[H]
    \centering
    \caption{Expected Return and Risk}
    \begin{tabular}{|c|c|c|}
        \hline
                     & Expected Return & Risk \\
        \hline
        Investment X & 10\%            & 7\%  \\
        Investment Y & 20\%            & 10\% \\
        Investment Z & 15\%            & 12\% \\
        \hline
    \end{tabular}
\end{table}
\textbf{Covariances:}
\begin{itemize}
    \item $Cov(X, Y) = 0.0032$
    \item $Cov(X, Z) = 0.0013$
    \item $Cov(Y, Z) = 0.0054$
\end{itemize}

\begin{table}[H]
    \centering
    \caption{Covariance Matrix}
    \begin{tabular}{|c|c|c|c|}
        \hline
        Covariance   & Investment X & Investment Y & Investment Z \\
        \hline
        Investment X & 0.0049       & 0.0032       & 0.0013       \\
        Investment Y & 0.0032       & 0.0100       & 0.0054       \\
        Investment Z & 0.0013       & 0.0054       & 0.0144       \\
        \hline
    \end{tabular}
\end{table}

\subsection{The Market Portfolio}
Using Solver in Excel, the weights for the Mean-Variance Efficient (Market) Portfolio are found to be:
$w_X = 0.0546$, $w_Y = 0.8424$, $w_Z = 0.1030$.
Since all weights are positive, this is a valid market portfolio.

\subsection*{Standard Deviation of the Market Portfolio}
\begin{align*}
    SD_P & = \sqrt{
        \begin{aligned}
             & 0.0546^2 \times 0.07^2 + 0.8424^2 \times 0.10^2 + 0.1030^2 \times 0.12^2 \\
             & + 2 \times 0.0546 \times 0.8424 \times 0.0032                            \\
             & + 2 \times 0.0546 \times 0.1030 \times 0.0013                            \\
             & + 2 \times 0.8424 \times 0.1030 \times 0.0054
        \end{aligned}
    }
\end{align*}

\[
    SD_P = 9.22\%
\]

\textbf{Market Expected Return ($E[R_m]$):}
\[
    E[R_m] = (0.0546 \times 10\%) + (0.8424 \times 20\%) + (0.1030 \times 15\%) = 18.94\%
\]

\textbf{Market Standard Deviation ($\sigma_m$):}
Calculated using the portfolio variance formula (sum of weighted variances and covariances):
\[
    \sigma_m = 9.22\% \implies \sigma_m^2 = 0.0922^2 \approx 0.0085
\]

\subsection{Calculating Covariance with the Market}
To find Beta, we calculate the covariance of each asset's returns with the market portfolio's returns. The covariance of X with the market is the weighted sum of the covariance of X with each component of the market.
\[
    Cov(R_X, R_m) = w_X Var(X) + w_Y Cov(X, Y) + w_Z Cov(X, Z)
\]
Note that $Cov(X,X)$ is simply the variance of X ($0.07^2$).
\[
    Cov(R_X, R_m) = 0.0546(0.07^2) + 0.8424(0.0032) + 0.1030(0.0013) \approx 0.0031
\]
Similarly, $Cov(R_Y, R_m) = 0.0092$ and $Cov(R_Z, R_m) = 0.0061$.

\subsection{Calculating Beta}
Using $\beta_i = \frac{Cov(R_i, R_m)}{\sigma_m^2}$ (where $\sigma_m^2 = 0.0922^2$):
\begin{itemize}
    \item $\beta_X = \frac{0.0031}{0.0085} = 0.36$
    \item $\beta_Y = \frac{0.0092}{0.0085} = 1.08$
    \item $\beta_Z = \frac{0.0061}{0.0085} = 0.72$
\end{itemize}
\textbf{Conclusion:} Asset Y has the highest expected return (20\%) and highest beta (1.08), while X has the lowest return (10\%) and lowest beta (0.36). This aligns with CAPM: higher systematic risk (beta) yields higher expected returns.


\noindent\rule{\linewidth}{1pt}
%-----------------------------------------------------------------

\section{Multifactor Models}

\subsection{Motivation}
So far, we have assumed that only one variable, the market portfolio, affects expected returns (as in CAPM). However, this may not be realistic for several reasons:
\begin{enumerate}
    \item \textbf{Other Single Factors:} It could be that another single factor, such as interest rates or the business cycle, affects expected returns for all risky assets.
    \item \textbf{Multiple Factors:} It is possible that a number of macroeconomic factors simultaneously determine expected returns.
\end{enumerate}

\subsection{Decomposition of Returns}
The return of any asset can be decomposed into two parts: an \textbf{expected} part and an \textbf{unanticipated} part. The unanticipated part consists of shocks attributable to risk factors and firm-specific shocks.
\[
    r_i = E[r_i] + \beta_i F + e_i
\]
Where:
\begin{itemize}
    \item $E[r_i]$: Expected return of asset $i$.
    \item $F$: Unanticipated shock to the risk factor ($E[F] = 0$).
    \item $\beta_i$: Sensitivity of asset $i$'s returns to the risk factor.
    \item $e_i$: Unanticipated firm-specific shock ($E[e_i] = 0$).
\end{itemize}

\textbf{Example (Single Factor):}
Consider Microsoft stock ($r_M$) where the single risk factor is GDP growth.
\[
    r_M = E[r_M] + \beta_M (GDP_{actual} - GDP_{expected}) + e_M
\]
Here, $(GDP_{actual} - GDP_{expected})$ is the unanticipated shock to the risk factor.

\subsection{Multifactor Model (K-Factors)}
We can extend the single index model to a model with $K$ factors.
\[
    r_i = E[r_i] + \beta_{i1}F_1 + \beta_{i2}F_2 + \dots + \beta_{iK}F_K + e_i
\]
Each $F_j$ represents the unanticipated shock to risk factor $j$, and $\beta_{ij}$ is the sensitivity to that factor.

\subsubsection{Two-Factor Example}
Consider a model with two risk factors: \textbf{GDP Growth} and \textbf{Inflation}.
\[
    r_M = E[r_M] + \beta_{M,1}(GDP_{act} - GDP_{exp}) + \beta_{M,2}(Inf_{act} - Inf_{exp}) + e_M
\]

\textbf{Numerical Example:}
\begin{itemize}
    \item Microsoft Expected Return ($E[r_M]$) = 10\%
    \item GDP: Expected = 4\%, Actual = 5\%. Sensitivity ($\beta_1$) = 1.
    \item Inflation: Expected = 6\%, Actual = 7\%. Sensitivity ($\beta_2$) = 0.4.
\end{itemize}
\[
    r_M = 10\% + 1(5\% - 4\%) + 0.4(7\% - 6\%) + e_M = 11.4\% + e_M
\]

\noindent\rule{\linewidth}{1pt}
%-----------------------------------------------------------------

\section{Arbitrage Pricing Theory (APT)}
\subsection{Learning Outcomes}
After this section, you will be able to:
\begin{itemize}
    \item Understand the Arbitrage Pricing Theory (APT).
    \item Use the APT to calculate expected returns.
\end{itemize}

\subsection{The Arbitrage Pricing Theory Model}
The APT is the most general asset pricing model. It relates expected returns and risk through factor betas. It posits that the expected return of asset $i$, $E[r_i]$, is given by:
\[
    E[r_i] = r_f + \beta_{i,1} RP_1 + \beta_{i,2} RP_2 + \dots + \beta_{i,K} RP_K
\]
Where:
\begin{itemize}
    \item $E[r_i]$: Expected return of asset $i$.
    \item $r_f$: Risk-free rate.
    \item $\beta_{i,j}$: Sensitivity of asset $i$'s returns to factor $j$.
    \item $RP_j$: Risk premium for factor $j$ (excess expected return over the risk-free rate).
\end{itemize}

\subsubsection{Comparison with CAPM}
\begin{itemize}
    \item \textbf{CAPM:} A specific case where the market portfolio is the only risk factor.
    \item \textbf{APT:} Allows for multiple risk factors. Like CAPM, it is a theoretical model.
\end{itemize}

\subsection{The Law of One Price and Arbitrage}
APT relies on the \textbf{Law of One Price}, which states that assets with the same future payoff must have the same current price.
\begin{itemize}
    \item \textbf{Arbitrage Opportunity:} A chance to make infinite profits with no risk and no net investment.
    \item If prices violate the Law of One Price, investors will exploit the arbitrage opportunity until prices correct.
\end{itemize}

\subsection{Determining Risk Premiums: An Example}
How do we determine the risk premium for each factor? We use \textbf{Factor Portfolios}. A factor portfolio tracks a specific source of macroeconomic risk ($\beta=1$ for that factor) and is uncorrelated with others ($\beta=0$ for others).

\textbf{Example Setup:}
\begin{itemize}
    \item Risk-free rate ($r_f$) = 3\%.
    \item \textbf{Portfolio G (GDP Factor):} $\beta_{GDP}=1, \beta_{Inf}=0, E[R_G]=10\%$.
    \item \textbf{Portfolio I (Inflation Factor):} $\beta_{GDP}=0, \beta_{Inf}=1, E[R_I]=13\%$.
    \item \textbf{Microsoft:} $\beta_{GDP}=1, \beta_{Inf}=0.4$.
\end{itemize}

\textbf{Replicating Portfolio Q:}
We form a portfolio Q to mimic Microsoft's sensitivities:
Weight in G ($w_G$) = 1.0; Weight in I ($w_I$) = 0.4; Weight in Risk-free ($w_{rf}$) = $1 - 1.0 - 0.4 = -0.4$.

\[
    E[R_Q] = (1 \times 10\%) + (0.4 \times 13\%) + (-0.4 \times 3\%) = 14\%
\]
To prevent arbitrage, Microsoft must also have an expected return of 14\%.

\textbf{Verification using APT Formula:}
\[
    E[r_M] = 3\% + 1(10\% - 3\%) + 0.4(13\% - 3\%) = 14\%
\]

\subsection{Security Market Plane and Relative Pricing}
With two factors, the relationship is a \textbf{Security Market Plane} (sensitivities on two axes, expected return on the third). Assets that do not lie on the SMP present arbitrage opportunities.

\textbf{Relative Pricing:} APT determines the price of an asset relative to the factor portfolios. It does not determine if the factor portfolios themselves are correctly priced (absolute pricing).

\noindent\rule{\linewidth}{1pt}
%-----------------------------------------------------------------

\section{Arbitrage and Factor Sensitivities}

\subsection{Learning Outcomes}
After this section, you will be able to:
\begin{itemize}
    \item Identify arbitrage opportunities.
    \item Solve for factor sensitivities given information on asset returns and unanticipated shocks to risk factors.
\end{itemize}

\subsection{Identifying Arbitrage Opportunities: An Example}
Consider a world with two time periods ($t=0$ and $t=1$) and two equally likely states in $t=1$: \textbf{Good} and \textbf{Bad} (50\% probability each).

\textbf{Stock Data:}
\begin{itemize}
    \item \textbf{Microsoft:} $P_0 = 25$. $P_1(\text{Good}) = 40$, $P_1(\text{Bad}) = 20$.
    \item \textbf{Intel:} $P_0 = 20$. $P_1(\text{Good}) = 60$, $P_1(\text{Bad}) = 0$.
    \item \textbf{Coke:} $P_0 = 22.5$. $P_1(\text{Good}) = 55$, $P_1(\text{Bad}) = 15$.
\end{itemize}

\textbf{Is there an arbitrage opportunity?}
Yes. We can construct a zero-cost portfolio with a guaranteed positive payoff.
\begin{itemize}
    \item \textbf{Strategy:} Buy 1 share of Coke, Short 0.5 share of Microsoft, Short 0.5 share of Intel.
    \item \textbf{Cost at $t=0$:}
          \[
              (1 \times 22.5) - (0.5 \times 25) - (0.5 \times 20) = 22.5 - 12.5 - 10 = 0
          \]
    \item \textbf{Payoff at $t=1$ (Good State):}
          \[
              (1 \times 55) - (0.5 \times 40) - (0.5 \times 60) = 55 - 20 - 30 = 5
          \]
    \item \textbf{Payoff at $t=1$ (Bad State):}
          \[
              (1 \times 15) - (0.5 \times 20) - (0.5 \times 0) = 15 - 10 - 0 = 5
          \]
\end{itemize}
\textbf{Result:} The portfolio costs nothing to set up and guarantees a payoff of 5 regardless of the state. This is an arbitrage opportunity.

\subsection{Using APT to Determine Correct Prices}
APT relies on relative pricing. Assuming Microsoft and Intel are correctly priced, we can determine the correct price of Coke (which implies Coke is currently mispriced at 22.5).

\subsubsection{Step 1: Identify Risk Factors}
Assume a single risk factor: the \textbf{Business Cycle}.
\begin{itemize}
    \item Value in Good State = 1.
    \item Value in Bad State = 0.
    \item Expected Value $E[F] = 0.5(1) + 0.5(0) = 0.5$.
\end{itemize}
\textbf{Unanticipated Shocks ($F - E[F]$):}
\begin{itemize}
    \item Good State: $1 - 0.5 = 0.5$.
    \item Bad State: $0 - 0.5 = -0.5$.
\end{itemize}

\subsubsection{Step 2: Determine Factor Sensitivities ($\beta$) and Expected Returns}
We use the single factor model equation:
\[
    r = E[r] + \beta (F_{\text{shock}})
\]
(Note: We assume no firm-specific risk for simplicity in this example).

\textbf{Microsoft Analysis:}
\begin{itemize}
    \item Return (Good): $\frac{40}{25} - 1 = 60\% = 0.60$.
    \item Return (Bad): $\frac{20}{25} - 1 = -20\% = -0.20$.
\end{itemize}
System of equations:
\begin{align*}
    0.60  & = E[r_M] + \beta_M(0.5)  \\
    -0.20 & = E[r_M] + \beta_M(-0.5)
\end{align*}
Solving this system:
\begin{itemize}
    \item Summing the equations: $0.40 = 2 E[r_M] \implies E[r_M] = 20\%$.
    \item Subtracting the second from the first: $0.80 = \beta_M(1) \implies \beta_M = 0.8$.
\end{itemize}

\textbf{Intel Analysis:}
\begin{itemize}
    \item Return (Good): $\frac{60}{20} - 1 = 200\% = 2.00$.
    \item Return (Bad): $\frac{0}{20} - 1 = -100\% = -1.00$.
\end{itemize}
System of equations:
\begin{align*}
    2.00  & = E[r_I] + \beta_I(0.5)  \\
    -1.00 & = E[r_I] + \beta_I(-0.5)
\end{align*}
Solving this system:
\begin{itemize}
    \item Summing the equations: $1.00 = 2 E[r_I] \implies E[r_I] = 50\%$.
    \item Subtracting the second from the first: $3.00 = \beta_I(1) \implies \beta_I = 3$.
\end{itemize}

\textit{Note: In the next section, we will use these values to compute the risk premium and the arbitrage-free price of Coke.}

\end{document}