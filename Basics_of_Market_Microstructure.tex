\documentclass[12pt]{article}
\usepackage[margin=1in]{geometry}
\usepackage{amsmath, amssymb, mathtools}
\usepackage{tikz}
\usepackage{lmodern}
\usepackage{hyperref}
\usepackage{caption}
\usepackage{float}
\usepackage{parskip}
\usepackage{tabularx}
\usepackage[utf8]{inputenc}
\usepackage{tgpagella}
\usepackage[T1]{fontenc}
\usepackage{array} % For better table features
\usepackage{booktabs}
\usepackage{tcolorbox}
\usepackage{enumitem}

\title{Basics of Market Microstructure}
\author{Zubin}
\date{\today}

\begin{document}
\maketitle

\tableofcontents

\newpage

\noindent\rule{\linewidth}{1pt}
%-----------------------------------------------------------------
\section{Markets and Limit Orders}

\subsection{Introduction to Trading Mechanics}
We now move on from the basics of financial statements and risk-return relationships to the mechanics of trading on exchanges. Transaction costs eat into investment returns, so it is important to understand how to measure these costs and manage orders to minimize them.

\subsection{What is an Order?}
An order is a set of instructions sent to an exchange (usually via a broker) to trade a financial security. Every order must specify:
\begin{enumerate}
    \item \textbf{Security:} The asset to trade, usually identified by a Ticker Symbol (e.g., AMZN for Amazon, F for Ford).
    \item \textbf{Side:} Whether to Buy or Sell.
    \item \textbf{Quantity:} The number of shares or units to trade.
\end{enumerate}
Orders may also include additional instructions, such as price limits, expiration terms, or exchange routing.

\subsection{Market Orders}
A \textbf{Market Order} is the simplest type of order, containing only the three required instructions (Security, Side, Quantity).
\begin{itemize}
    \item \textbf{Definition:} An order to execute the trade immediately at the best currently available price.
    \item \textbf{Pros:} Execution is almost certain.
    \item \textbf{Cons:} Price is uncertain. You accept whatever the market offers, which might be unfavorable in volatile conditions.
\end{itemize}
\textbf{Example:} A market order to buy 100 shares of Twitter guarantees you get the shares, but the price is determined by the market at the moment of execution.

\subsection{Limit Orders}
A \textbf{Limit Order} allows the trader to specify the price at which they are willing to trade. This specified price is called the \textbf{Limit Price}.
\begin{itemize}
    \item \textbf{Buy Limit Order:} Sets a \textbf{maximum} purchase price. The trade executes only at the limit price or lower.
    \item \textbf{Sell Limit Order:} Sets a \textbf{minimum} selling price. The trade executes only at the limit price or higher.
\end{itemize}
\textbf{Example:} If you place a limit order to buy Twitter at \$19, you ensure you never pay more than \$19. If the market price is \$19.50, the order will not execute.

\subsection{Summary: Market vs. Limit Orders}
\begin{itemize}
    \item \textbf{Market Orders:} Execution is certain; Price is uncertain.
    \item \textbf{Limit Orders:} Price is controlled (certain not to be worse than limit); Execution is uncertain.
\end{itemize}
Most modern exchanges are \textbf{Electronic Limit Order Markets}, where limit and market orders interact to facilitate trading.

\noindent\rule{\linewidth}{1pt}
%-----------------------------------------------------------------
\section{Limit Order Book}

\subsection{Introduction}
In this section, we examine how limit orders enter the order book and await execution. We will define key concepts such as the best bid and ask prices, the bid-ask spread, and market depth. Furthermore, we will observe how market orders and marketable limit orders interact with the order book to generate trades.

\subsection{Definitions}
\begin{itemize}
    \item \textbf{Best Ask Price:} The lowest price at which someone is willing to sell shares.
    \item \textbf{Best Bid Price:} The highest price at which someone is willing to buy shares.
    \item \textbf{Bid-Ask Spread:} The difference between the best ask price and the best bid price.
    \item \textbf{Depth:} The number of shares available for trading at a specific price level.
\end{itemize}

\subsection{Order Book Dynamics: An Example}
Consider the limit order book for a hypothetical stock, XYZ Incorporated. Initially, the order book is empty.

\subsubsection{Step 1: Arrival of Passive Orders}
A trader sends an order (\textbf{S1}) to sell 150 shares of XYZ with a limit price of \$40.
\begin{itemize}
    \item This order sits on the sell side.
    \item \textbf{Current Best Ask:} \$40.
\end{itemize}

A few seconds later, a buy order (\textbf{B1}) for 225 shares at \$39.95 enters the market.
\begin{itemize}
    \item S1 is not willing to sell below \$40, and B1 is not willing to pay more than \$39.95. They do not execute.
    \item \textbf{Current Best Bid:} \$39.95.
    \item \textbf{Bid-Ask Spread:} $40 - 39.95 = \$0.05$.
\end{itemize}

These orders (S1 and B1) are referred to as \textbf{Standing} or \textbf{Passive Limit Orders}. They provide liquidity and face execution uncertainty (no guarantee they will be filled).

\subsubsection{Step 2: Improving the Bid}
Another buy order (\textbf{B2}) arrives for 500 shares at \$39.97.
\begin{itemize}
    \item B2 does not match S1 (\$40), but it offers a higher price than B1.
    \item \textbf{New Best Bid:} \$39.97.
    \item \textbf{New Spread:} $40 - 39.97 = \$0.03$.
\end{itemize}

\textbf{Current Depth:}
\begin{itemize}
    \item Depth at Best Ask (\$40): 150 shares.
    \item Depth at Best Bid (\$39.97): 500 shares.
\end{itemize}

\subsubsection{Step 3: Market Order Execution}
A market order (\textbf{S2}) to sell 300 shares arrives.
\begin{itemize}
    \item Market orders trade at the current best available prices.
    \item S2 interacts with the best bid (B2 at \$39.97).
    \item \textbf{Execution:} 300 shares trade at \$39.97.
    \item \textbf{Result:} S2 is filled completely. B2 has 200 shares remaining ($500 - 300$).
    \item The Best Bid price remains \$39.97, but depth has decreased to 200.
\end{itemize}

\subsubsection{Step 4: Marketable Limit Order}
A sell order (\textbf{S3}) arrives for 250 shares with a limit price of \$39.95.
\begin{itemize}
    \item S3 is willing to sell at \$39.95 or higher. The best bid is \$39.97 (higher than limit), so it executes immediately.
    \item \textbf{Trade 1:} 200 shares execute against the remainder of B2 at \$39.97. (B2 is now filled).
    \item S3 still needs to sell 50 shares. The next best bid is B1 at \$39.95.
    \item \textbf{Trade 2:} 50 shares execute against B1 at \$39.95.
    \item \textbf{Result:} S3 is filled. B1 has 175 shares remaining ($225 - 50$).
\end{itemize}

\textbf{Marketable Limit Orders (Active Orders):}
S3 is an example of a marketable limit order. Like a market order, execution was immediate (no uncertainty). However, price was uncertain (executed at multiple prices: \$39.97 and \$39.95), though never worse than the limit price.

\subsubsection{Final State}
\begin{itemize}
    \item \textbf{Best Ask:} \$40 (S1, 150 shares).
    \item \textbf{Best Bid:} \$39.95 (Remainder of B1, 175 shares).
    \item \textbf{Spread:} \$0.05.
\end{itemize}

\end{document}