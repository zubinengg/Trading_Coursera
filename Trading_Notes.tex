\documentclass[14pt]{article}
\usepackage[margin=1in]{geometry}
\usepackage{amsmath, amssymb, mathtools}
\usepackage{tikz}
\usepackage{lmodern}
\usepackage{hyperref}
\usepackage{caption}
\usepackage{float}
\usepackage{parskip}
\usepackage{tabularx}
\usepackage[utf8]{inputenc}
\usepackage{tgpagella}
\usepackage[T1]{fontenc}
\usepackage{array} % For better table features

\usepackage[utf8]{inputenc}
\usepackage[margin=1in]{geometry}
\usepackage{booktabs}
\usepackage{tcolorbox}
\usepackage{enumitem}

\title{Introduction to Accounting: Lecture Notes}
\author{}
\date{}

\begin{document}

\maketitle

\tableofcontents

\newpage

\section{Definition of Accounting}
\textbf{Accounting} is the formal collection, aggregation, analysis, and reporting of financial and non-financial data about a company to various end users.

\begin{itemize}
    \item \textbf{Mandated Reporting:} Public companies are required to release data periodically (typically quarterly and annually).
    \item \textbf{End Users:} Data is provided to both internal parties (managers) and external parties (investors, creditors).
\end{itemize}

\section{The Three Types of Accounting}
The field of accounting is broadly categorized into three distinct branches based on the target audience and purpose:

\begin{table}[h]
    \centering
    \begin{tabular}{@{}lll@{}}
        \toprule
        \textbf{Type} & \textbf{Primary Audience}   & \textbf{Purpose \& Characteristics}                   \\ \midrule
        Financial     & External (Investors, Banks) & Assess performance, health, and credit risk.          \\
        Managerial    & Internal (Management)       & Decision making: pricing, product mix, and budgeting. \\
        Tax           & Tax Authorities (IRS)       & Compliance and estimation of tax liabilities.         \\ \bottomrule
    \end{tabular}
    \caption{Comparison of Accounting Branches}
\end{table}

\subsection{Financial Accounting}
\begin{itemize}
    \item \textbf{Primary Purpose:} To inform outsiders (equity investors, investment banks) for valuation and credit risk assessment.
    \item \textbf{Secondary Purpose:} To monitor managers and ensure responsible use of company resources.
    \item \textbf{Compliance:} Reports must follow specific rules and are audited by external parties for accuracy.
\end{itemize}

\subsection{Managerial Accounting}
\begin{itemize}
    \item \textbf{Purpose:} Capacity planning, determining production volumes, and evaluating product profitability.
    \item \textbf{Format:} No fixed formats; reports are not audited.
\end{itemize}

\section{Accrual Accounting}


\subsection{Meaning of Accrual Basis of Accounting}

Under the \textbf{accrual basis of accounting}:
A key element of financial accounting is the concept of \textbf{Accruals}. Financial statements are prepared on an accrual basis rather than a cash basis.
\begin{itemize}
    \item Income is recognized when it is earned, irrespective of when cash is received.
    \item Expenses are recognized when they are incurred, irrespective of when cash is paid.
\end{itemize}

Thus, transactions are recorded at the time when the economic activity occurs, not when cash flows take place.

\subsection{Meaning of Cash Basis of Accounting}

Under the \textbf{cash basis of accounting}:
\begin{itemize}
    \item Income is recorded only when cash is actually received.
    \item Expenses are recorded only when cash is actually paid.
\end{itemize}

This method ignores outstanding receivables, payables, and other non-cash adjustments.

\subsection{Why Accrual Basis is Preferred for Financial Statements}

\subsubsection{True and Fair View}

Accrual accounting follows the matching principle, whereby expenses are matched with the revenues earned during the same accounting period. This results in a true and fair view of financial performance.

\subsubsection{Correct Measurement of Profit}

Cash receipts and payments may relate to different accounting periods. Accrual accounting ensures that only the income and expenses pertaining to the current period are considered.

\subsubsection{Better Assessment of Financial Position}

Assets such as receivables and liabilities such as outstanding expenses are recognized, providing a realistic picture of the financial position of the entity.

\subsubsection{Comparability and Consistency}

Financial statements prepared on an accrual basis are comparable across different periods and entities, facilitating better decision-making by users.

\subsection{Key Principles}
Transactions are recognized when the \textit{economic event} occurs, not necessarily when cash changes hands.
\begin{enumerate}
    \item \textbf{Revenue Recognition:} Revenue is reported when earned (delivery of product/service), measurable, and collection is reasonably certain.
    \item \textbf{Matching Principle:} Expenses related to revenues must be recognized in the same period as the revenue itself.
\end{enumerate}

\subsection{The Fundamental Difference}
The distinction between earnings and cash flow is often summarized by the adage:
\begin{equation*}
    \text{Cash Flow} = \text{Fact} \quad \longleftrightarrow \quad \text{Earnings} = \text{Opinion}
\end{equation*}

\subsection{Pros and Cons of Accrual Accounting}
\begin{itemize}
    \item \textbf{Advantages:} Provides timelier, decision-relevant information by reflecting economic reality.
    \item \textbf{Disadvantages:} Relies on managerial judgment and estimates, making it potentially less reliable than raw cash flow data.
\end{itemize}


\subsection{Summary of Financial Statements}
Accrual accounting directly results in the creation of two primary statements:
\begin{itemize}
    \item \textbf{Income Statement:} Also known as the Profit and Loss (P\&L) statement.
    \item \textbf{Balance Sheet:} Reflects the net financial position of the company.
\end{itemize}

\noindent\rule{\linewidth}{1pt}

\section{Introduction to the Balance Sheet}
The \textbf{Balance Sheet} captures the financial position of a company as of a \textit{particular date} (e.g., the end of a quarter or the end of a fiscal year). It acts as a snapshot in time, showing what a company owns and what it owes.

\subsection{The Accounting Equation}
The balance sheet derives its name from the fact that it must always balance. This relationship is expressed through the fundamental accounting equation:

\begin{equation}
    \text{Assets} = \text{Liabilities} + \text{Shareholders' Equity}
\end{equation}

\begin{itemize}
    \item \textbf{Assets:} Represent how the company \textit{uses} its resources.
    \item \textbf{Liabilities \& Equity:} Represent \textit{where} the company gets its resources from.
\end{itemize}

\subsection{Components of the Balance Sheet}

\subsubsection{Assets}
Assets are resources owned by the company used to generate future economic benefits (higher cash inflows or lower cash outflows).
\begin{itemize}
    \item \textbf{Classification:}
          \begin{itemize}
              \item \textbf{Current Assets:} Expected to be converted to cash, sold, or consumed within \textbf{one year}. (Examples: Cash, equivalents, inventory).
              \item \textbf{Non-current Assets:} Expected to be realized after one year. (Examples: Property, Plant, and Equipment [PP\&E], Patents, Trademarks).
          \end{itemize}
\end{itemize}

\subsubsection{Liabilities}
Liabilities represent the company’s economic obligations to outsiders.
\begin{itemize}
    \item \textbf{Classification:}
          \begin{itemize}
              \item \textbf{Current Liabilities:} Obligations expected to be paid within \textbf{one year}. (Examples: Accounts payable, short-term borrowing).
              \item \textbf{Non-current Liabilities:} Obligations expected to be settled after one year. (Example: Long-term debt).
          \end{itemize}
\end{itemize}

\subsubsection{Shareholders' Equity}
This represents the owners' claims on the total assets of the company.
\begin{itemize}
    \item \textbf{Contributed Capital:} Investments made by the owners.
    \item \textbf{Retained Earnings:} The aggregate undistributed profits of the company over time.
\end{itemize}

\subsection{Key Characteristics}
\begin{itemize}
    \item \textbf{Historical Cost:} Items are recorded at their original purchase price. They are generally \textbf{not} updated to reflect current market values.
    \item \textbf{Presentation:}
          \begin{itemize}
              \item Typically, Assets are on the left (or top), while Liabilities and Equity are on the right (or bottom).
              \item Formats may vary internationally based on local accounting standards.
          \end{itemize}
\end{itemize}

\noindent\rule{\linewidth}{1pt}


\section{Overview of Assets}
Assets represent how a company uses its resources to generate future economic benefits. On the balance sheet, assets are categorized based on their liquidity and expected lifespan.

\subsection{Current Assets}
Current assets are resources expected to be sold, converted to cash, or consumed within \textbf{one year}.

\subsubsection{Cash and Cash Equivalents}
This is the most liquid category. It includes:
\begin{itemize}
    \item Currency, coins, and petty cash.
    \item Checks received but not yet deposited.
    \item Checking/savings accounts and money market accounts.
    \item Short-term, highly liquid investments with maturities of \textbf{three months or less} at the time of purchase.
    \item \textit{Example:} Amazon (2015) held \$15.89 billion.
\end{itemize}

\subsubsection{Marketable Securities}
Investments in financial securities like stocks and bonds. While easily convertible to cash, they are considered less liquid than cash equivalents.
\begin{itemize}
    \item \textit{Example:} Amazon (2015) held \$3.92 billion.
\end{itemize}

\subsubsection{Inventories}
Includes physical goods at various stages of production:
\begin{itemize}
    \item \textbf{Raw Materials:} Unprocessed materials.
    \item \textbf{Work in Progress (WIP):} Partially finished goods.
    \item \textbf{Finished Goods:} Products ready for sale.
    \item \textit{Note:} Service industries may only have supplies, while retailers like Amazon hold massive finished goods (\$10.24 billion in 2015).
\end{itemize}

\subsubsection{Accounts Receivable}
Money owed to the company by its clients/customers. This occurs when a company extends credit by delivering a product or service before receiving payment.
\begin{itemize}
    \item Represents a legal obligation from the client to the company.
    \item \textit{Example:} Amazon (2015) had \$6.42 billion due from customers, vendors, and sellers.
\end{itemize}

\subsection{Non-Current Assets}
Non-current assets are long-term investments and resources that are expected to provide value for more than one year.

\subsubsection{Property, Plant, and Equipment (PP\&E)}
The largest component of non-current assets for most physical businesses.
\begin{itemize}
    \item \textbf{Meaning of "Plant":} Refers to industrial facilities (e.g., factories, warehouses).
    \item \textbf{Tangible Items:} Land, buildings, machinery, office furniture, vehicles, and fixtures.
    \item \textbf{Technology:} Servers, networking equipment, and internal-use software.
    \item \textit{Example:} Amazon (2015) held \$21.84 billion in PP\&E (including capital leases).
\end{itemize}

\subsubsection{Intangible Assets}
Non-physical assets that contribute to a company's future value.
\begin{itemize}
    \item \textbf{Intellectual Property:} Patents, trademarks, and copyrights.
    \item \textbf{Goodwill:} The value of a company's brand recognition and reputation (Amazon 2015: \$3.76 billion).
          \begin{itemize}
              \item \textit{Calculation:} Recorded only during an acquisition.
              \item $\text{Goodwill} = \text{Purchase Price} - (\text{Fair Value of Assets} - \text{Fair Value of Liabilities})$
          \end{itemize}
\end{itemize}

\subsection{Total Asset Summary: Case Study Amazon (2015)}
The total value of assets must be funded by either liabilities or equity.
\begin{itemize}
    \item \textbf{Total Current Assets:} \$36.47 billion
    \item \textbf{Total Non-Current Assets:} \$28.97 billion
    \item \textbf{Total Assets:} \$65.44 billion
\end{itemize}

\noindent\rule{\linewidth}{1pt}

\section{Overview of Liabilities}
Liabilities represent where the company gets some of its resources from, specifically the obligations to outsiders. Like assets, they are classified based on when they are due.

\subsection{Current Liabilities}
Obligations that are due within the next \textbf{one year}.

\subsubsection{Accounts Payable}
The amount the company owes its vendors and suppliers for products and services that have already been provided or delivered on credit.
\begin{itemize}
    \item \textit{Example:} Amazon (2015) owed \$20.40 billion.
\end{itemize}

\subsubsection{Accrued Expenses}
Expenses the company has incurred but has not yet received an invoice for (e.g., wages, interest, utilities).
\begin{itemize}
    \item \textit{Example:} Amazon (2015) had \$10.38 billion (related to unredeemed gift cards, leases, and asset retirement obligations).
\end{itemize}

\subsubsection{Unearned Revenue}
Money paid by customers in advance for products or services. It remains a liability until the company delivers the promised product/service.
\begin{itemize}
    \item \textit{Example:} Amazon (2015) had \$3.12 billion (primarily Prime memberships and AWS prepayments).
\end{itemize}

\textbf{Total Current Liabilities (Amazon 2015):} \$33.90 billion.

\subsection{Non-Current Liabilities}
Obligations due to be paid \textbf{after one year}.

\subsubsection{Long-Term Debt}
Includes loans and bonds.
\begin{itemize}
    \item \textit{Example:} Amazon (2015) had \$8.23 billion.
\end{itemize}

\subsubsection{Other Long-Term Liabilities}
Includes long-term lease obligations, tax contingencies, and deferred tax liabilities.
\begin{itemize}
    \item \textit{Example:} Amazon (2015) had \$9.93 billion.
\end{itemize}

\textbf{Total Non-Current Liabilities (Amazon 2015):} \$18.16 billion.

\subsection{Total Liability Summary: Case Study Amazon (2015)}
\begin{itemize}
    \item \textbf{Total Current Liabilities:} \$33.90 billion
    \item \textbf{Total Non-Current Liabilities:} \$18.16 billion
    \item \textbf{Total Liabilities:} \$52.06 billion
\end{itemize}

Comparing this to Total Assets (\$65.44 billion), the remaining balance (\$13.37 billion) represents the resources provided by Shareholders' Equity.

\noindent\rule{\linewidth}{1pt}
%%-----------------------------------------------------------------

\section{Overview of Shareholders' Equity}
Shareholders' Equity (also referred to as Net Worth) represents the total equity interest of all shareholders. It is the residual value of assets after deducting liabilities.

\subsection{Components of Shareholders' Equity}

\subsubsection{Common Stock}
Represents the primary ownership of the company.
\begin{itemize}
    \item \textbf{Voting Rights:} Shareholders can select the Board of Directors.
    \item \textbf{Par Value (Capital Stock):} A nominal value assigned to each share (e.g., \$0.01). This represents the legal capital.
    \item \textbf{Accounting Split:} When Common Stock is issued, the proceeds are split between "Capital Stock" (par value) and "Additional Paid-in Capital" (excess over par).
    \item \textit{Example:} Amazon (2015) had 500 million shares outstanding at \$0.01 par value (\$5 million total).
\end{itemize}

\subsubsection{Additional Paid-in Capital}
The amount of capital the company raises above the par value (Capital Stock) of the shares when they are sold to the public.
\begin{itemize}
    \item \textit{Example:} Amazon (2015) had \$13.39 billion.
\end{itemize}

\subsubsection{Treasury Stock}
Common stock that the company has bought back from shareholders or authorized but never sold.
\begin{itemize}
    \item Recorded at cost and deducted from total equity (contra-equity).
    \item These shares have no voting rights and receive no dividends.
    \item \textit{Example:} Amazon (2015) held \$1.84 billion.
\end{itemize}

\subsubsection{Preferred Stock}
A class of stock that has priority over common stock regarding dividends.
\begin{itemize}
    \item \textbf{Priority:} Dividends must be paid to preferred stockholders before common stockholders.
    \item \textbf{No Voting Rights:} Preferred stockholders generally do not vote.
    \item \textit{Example:} Amazon (2015) had not issued any preferred stock (\$0).
\end{itemize}

\subsubsection{Retained Earnings}
The aggregate of undistributed profits across all years.
\begin{itemize}
    \item Represents profits not paid out as dividends but reinvested in the company.
    \item \textit{Example:} Amazon (2015) had \$2.55 billion.
\end{itemize}

\subsubsection{Other Accumulated Comprehensive Loss}
Includes unrealized gains and losses from investments in cash equivalents and marketable securities.
\begin{itemize}
    \item \textit{Example:} Amazon (2015) had a loss of \$0.72 billion.
\end{itemize}

\subsection{Total Equity Summary: Case Study Amazon (2015)}
Adding the components together yields the total shareholders' equity.
\begin{itemize}
    \item \textbf{Total Shareholders' Equity:} \$13.38 billion.
\end{itemize}

\subsection{Balance Sheet Equation Check}
\begin{equation*}
    \text{Liabilities} (\$52.06\,\text{B}) + \text{Equity} (\$13.38\,\text{B}) = \text{Total Assets} (\$65.44\,\text{B})
\end{equation*}
This confirms that Amazon's uses of resources (Assets) exactly match its sources of resources (Liabilities + Equity).

\noindent\rule{\linewidth}{1pt}

\section{Introduction to the Income Statement}
The \textbf{Income Statement} (also known as the Profit and Loss statement or P\&L) reports a company's operating results for a specific period. It primarily details revenues and costs.
\begin{itemize}
    \item \textbf{Amazon Terminology:} Referred to as "Consolidated Statements of Operations".
    \item \textbf{Purpose:} Tells analysts and investors how profitable a company is.
    \item \textbf{Basic Formula:}
          \begin{equation*}
              \text{Net Income (Profit/Loss)} = \text{Revenues} - \text{Expenses}
          \end{equation*}
\end{itemize}

\subsection{Expensing vs. Capitalizing}
Assets and expenses both require the use of company resources, but they are treated differently based on \textit{when} the benefits are realized.

\subsubsection{Expensing}
If the benefits of the resource use result in the \textbf{same period} as the use itself:
\begin{itemize}
    \item The use is categorized as an \textbf{Expense}.
    \item The entire amount appears on that period's Income Statement.
\end{itemize}

\subsubsection{Capitalizing}
If the benefits are expected to accrue in \textbf{future time periods}:
\begin{itemize}
    \item The use is \textbf{Capitalized}.
    \item The amount is shown as an \textbf{Asset} on the Balance Sheet (e.g., PP\&E, purchased patents).
\end{itemize}

\subsection{Converting Capitalized Costs to Expenses}
Capitalized expenditures are eventually charged as expenses to the Income Statement through two processes:
\begin{enumerate}
    \item \textbf{Amortization:} The process of allocating an asset's value to future periods when its benefits are earned (typically over the asset's useful life). It is a periodic charge.
          \begin{itemize}
              \item \textit{Depreciation:} The specific terminology used for amortizing Property, Plant, and Equipment (PP\&E).
          \end{itemize}
    \item \textbf{Impairment:} A write-down of an asset's value to its current fair market value due to substantial value destruction (by market forces or nature). It is a one-time event.
\end{enumerate}

\subsection{Revenues (Sales)}
Often called the "Top Line" because it is the first item on the Income Statement.
\begin{itemize}
    \item Includes income only from \textbf{primary lines of business} (excludes investment income).
    \item \textbf{"Net" Sales:} Revenues minus returns and discounts.
    \item \textbf{Amazon (2015) Case Study:}
          \begin{itemize}
              \item Net Product Sales (own inventory): \$76.27 billion.
              \item Net Service Sales (3rd party commissions, AWS): \$27.74 billion.
              \item \textbf{Total Net Sales:} \$107.01 billion.
          \end{itemize}
\end{itemize}

\noindent\rule{\linewidth}{1pt}
%%-----------------------------------------------------------------
\end{document}